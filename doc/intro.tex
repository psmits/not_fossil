\documentclass[12pt,letterpaper]{article}

\usepackage{amsmath, amsthm, amsfonts, amssymb}
\usepackage{microtype, parskip, graphicx}
\usepackage[comma,numbers,sort&compress]{natbib}
\usepackage{lineno}
\usepackage{longtable}
\usepackage{docmute}
\usepackage{caption, subcaption, multirow, morefloats, rotating}
\usepackage{wrapfig}
\usepackage{hyperref}

\frenchspacing

\begin{document}

\section{Introduction}

% to find fossils we first have to find rocks (a geological unit)
% only some rocks (units) are associated with fossils
% these rocks record specific environments
% or perception of diversity is shaped by which environments are preserved
% what geological properties affect the observed alpha diversity of a geological unit?

% what happens if we lose good rocks/certain environments?
%   e.g. sea level drop changes near-shore habitats dramatically
%   if a species disappears, is it because 
%     they are extinct?
%   we couldn't sample its habitat?
% perhaps the features of rocks that bear fossils during those intervals will help?

% we have two major questions
%   do geological units during periods of low sea level have ``different'' diversity (than expected)?
%     doesn't necessarily have be lower
%     poor estimation can mean that unit is different than we would expect based on our model
%   do geological units during periods of low seal level have different assocations with geological features (than other during times)?
%     every time point is inherently different
%     we're more interested in a systematic difference (sign switch, loss of effect, etc.)

% our approach is to estimate unit diversity as a function of the geological properties of the unit
% we use a hierarchical time series model which allowing the effects of covariates to vary over time

\end{document}
